% This is the Duke University Statistical Science LaTeX thesis template.
% It has been adapted from the Reed College LaTeX thesis template. The
% adaptation was done by Mine Cetinkaya-Rundel (MCR). Some of the comments
% that are specific to Reed College have been removed.
%
% Most of the work on the original Reed College document class and template
% was done by Sam Noble (SN). Later comments etc. by Ben Salzberg (BTS).
% Additional restructuring and APA support by Jess Youngberg (JY).
%
% See https://www.reed.edu/cis/help/latex/ for help. There are a
% great bunch of help pages there, with notes on
% getting started, bibtex, etc. Go there and read it if you're not
% already familiar with LaTeX.
%
% Any line that starts with a percent symbol is a comment.
% They won't show up in the document, and are useful for notes
% to yourself and explaining commands.
% Commenting also removes a line from the document;
% very handy for troubleshooting problems. -BTS

%%
%% Preamble
%%
% \documentclass{<something>} must begin each LaTeX document
\documentclass[12pt,oneside]{dukestatscithesis}
% Packages are extensions to the basic LaTeX functions. Whatever you
% want to typeset, there is probably a package out there for it.
% Chemistry (chemtex), screenplays, you name it.
% Check out CTAN to see: http://www.ctan.org/
%%
\usepackage{graphicx,latexsym}
\usepackage{amsmath}
\usepackage{amssymb,amsthm}
\usepackage{longtable,booktabs,setspace}
\usepackage{chemarr} %% Useful for one reaction arrow, useless if you're not a chem major
\usepackage[hyphens]{url}
% Added by CII
\usepackage{hyperref}
\usepackage{lmodern}
\usepackage{float}
\floatplacement{figure}{H}
% End of CII addition
\usepackage{rotating}

% Next line commented out by CII
%%% \usepackage{natbib}
% Comment out the natbib line above and uncomment the following two lines to use the new
% biblatex-chicago style, for Chicago A. Also make some changes at the end where the
% bibliography is included.
%\usepackage{biblatex-chicago}
%\bibliography{thesis}


% Added by CII (Thanks, Hadley!)
% Use ref for internal links
\renewcommand{\hyperref}[2][???]{\autoref{#1}}
\def\chapterautorefname{Chapter}
\def\sectionautorefname{Section}
\def\subsectionautorefname{Subsection}
% End of CII addition

% Added by CII
\usepackage{caption}
\captionsetup{width=5in}
% End of CII addition

% \usepackage{times} % other fonts are available like times, bookman, charter, palatino


% To pass between YAML and LaTeX the dollar signs are added by CII
\title{EAS 499 Final Thesis}
\author{Ali Kozlu}
% The month and year that you submit your FINAL draft TO THE LIBRARY (May or December)
\date{Dec 2018}
\advisor{Professor Marcus Mitchell}
\institution{University Of Pennsylvania}
\degree{Bachelor of Science in Computer and Cognitive Science}
%\committeememberone{}
%\committeemembertwo{}
%\dus{}
%If you have two advisors for some reason, you can use the following
% Uncommented out by CII
% End of CII addition

%%% Remember to use the correct department!
\department{Department of Computer and Cognitive Science}

% Added by CII
%%% Copied from knitr
%% maxwidth is the original width if it's less than linewidth
%% otherwise use linewidth (to make sure the graphics do not exceed the margin)
\makeatletter
\def\maxwidth{ %
  \ifdim\Gin@nat@width>\linewidth
    \linewidth
  \else
    \Gin@nat@width
  \fi
}
\makeatother

\renewcommand{\contentsname}{Table of Contents}
% End of CII addition

\setlength{\parskip}{0pt}

% Added by CII
  %\setlength{\parskip}{\baselineskip}
  \usepackage[parfill]{parskip}

\providecommand{\tightlist}{%
  \setlength{\itemsep}{0pt}\setlength{\parskip}{0pt}}

\Acknowledgements{

}

\Dedication{

}

\Preface{

}

\Abstract{

}

	\usepackage{amsmath}
	\usepackage{float}
	\usepackage{booktabs}
	\usepackage{longtable}
	\usepackage{array}
	\usepackage{multirow}
	\usepackage[table]{xcolor}
	\usepackage{wrapfig}
	\usepackage{float}
	\usepackage{colortbl}
	\usepackage{pdflscape}
	\usepackage{tabu}
	\usepackage{threeparttable}
	\usepackage{threeparttablex}
	\usepackage[normalem]{ulem}
	\usepackage{makecell}
	\usepackage{xcolor}
% End of CII addition
%%
%% End Preamble
%%
%

\usepackage{amsthm}
\newtheorem{theorem}{Theorem}[chapter]
\newtheorem{lemma}{Lemma}[chapter]
\theoremstyle{definition}
\newtheorem{definition}{Definition}[chapter]
\newtheorem{corollary}{Corollary}[chapter]
\newtheorem{proposition}{Proposition}[chapter]
\theoremstyle{definition}
\newtheorem{example}{Example}[chapter]
\theoremstyle{definition}
\newtheorem{exercise}{Exercise}[chapter]
\theoremstyle{remark}
\newtheorem*{remark}{Remark}
\newtheorem*{solution}{Solution}
\begin{document}

% Everything below added by CII
  \maketitle

\frontmatter % this stuff will be roman-numbered
\pagestyle{empty} % this removes page numbers from the frontmatter



  \hypersetup{linkcolor=black}
  \setcounter{tocdepth}{2}
  \tableofcontents

  \listoftables

  \listoffigures



\mainmatter % here the regular arabic numbering starts
\pagestyle{fancyplain} % turns page numbering back on

\chapter{thesisdowndss::thesis\_epub:
default}\label{thesisdowndssthesis_epub-default}

Placeholder

\chapter{This chunk ensures that the thesisdowndss package
is}\label{this-chunk-ensures-that-the-thesisdowndss-package-is}

Placeholder

\section{Review of Previous Work}\label{review-of-previous-work}

\section{Ramer-Douglas-Peucker
algorithm}\label{ramer-douglas-peucker-algorithm}

\section{Application of RDP Algorithm to Play
Data}\label{application-of-rdp-algorithm-to-play-data}

\chapter{This chunk ensures that the thesisdowndss package
is}\label{this-chunk-ensures-that-the-thesisdowndss-package-is-1}

Placeholder

\section{The Rub Concept}\label{the-rub-concept}

\section{Determining Route
Intersection}\label{determining-route-intersection}

\section{Identfying Slant-Flat
Concept}\label{identfying-slant-flat-concept}

\section{Limitations}\label{limitations}

\section{Future Works}\label{future-works}

\chapter*{Conclusion}\label{conclusion}
\addcontentsline{toc}{chapter}{Conclusion}

The present work hopes to add to extensive research that has been
conducted into the prediction of tennis matches. We have looked at how
ensemble models can be used for tennis match predictor domain. Using
110008 tennis matches played between 2004-2017, we have devised a
learning set that consists of seven features based on research by Sipko
and Knottenbelt, using their proposed methods of weighing historical
matches. We then introduced stacked generalization as a novel tennis
prediction model. Based on our knowledge and research, no one has used
an ensembling technique to combine information from multiple predictive
models to generate a new tennis match model. We investigated whether the
performance of multiple ensemble models can be replicated in the domain
of tennis match prediction. Although all of our proposed ensemble models
perform with similar accuracy compared to state of the art models
proposed by other researchers, we argue that tennis models cannot yet
achieve a positive return on investment on traditional pre-game betting
markets against bookmakers. We then turn our attention to betting
exchange markets and propose a new approach that uses model class
probabilities to make a profit by predicting future tennis market
movements. An example from Wimbledon 2018 is given to show why we think
using class probabilities can give us new avenues of profit. Although we
have only tested this new framework fully during the ATP Finals in
November, we will continue our work to create a framework for betting on
exchange markets before the first Grand Slam Tournament of 2019 season.

\chapter*{References}\label{references}
\addcontentsline{toc}{chapter}{References}

Placeholder


% Index?

\end{document}
